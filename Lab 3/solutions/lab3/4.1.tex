When PC4 tries to ping PC1 in the first scenario, it tries to send an ARP request to get the physical address of PC1. However, PC1 is not directly reachable by PC4. The ARP request message reaches the Cisco router, which in turn will forward the request to its other network (because proxy ARP is enabled). A response to the ARP request eventually reaches back to PC4. This can be seen by inspecting the packets at PC1 and PC4 (the ARP request PC4 sent can be seen at PC1 as well).

\lstinputlisting{traces/4.5.PC1.txt}
\lstinputlisting{traces/4.5.PC4.txt}

PC4's ARP table also contains PC1's MAC address as a result.
\lstinputlisting{traces/4.5.ARP.PC4.txt}

PC4 can ping PC1 without problems after ARP is resolved (normal ICMP packets were captured over the network in all hosts, which can be verified in these files: "4.5.PC1.pcapng", "4.5.PC2.pcapng", "4.5.PC1.pcap"). \\ \\
For the second scenario (where proxy ARP is disabled), the ping is not succesful. PC4 will once again try to send an ARP request for PC1's MAC address. The ARP request reaches the router, but the packet will not be forwarded since proxy ARP is disabled. PC4 receives no reply to the ARP request, and will say that the host is unreachable.
\lstinputlisting{traces/4.7.ping.PC4.txt}
\lstinputlisting{traces/4.7.PC4.txt}