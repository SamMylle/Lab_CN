\textit{The OSPF part of the LS Update packet from question 4.C.4.}
\begin{lstlisting}
Open Shortest Path First
    OSPF Header
        Version: 2
        Message Type: LS Update (4)
        Packet Length: 76
        Source OSPF Router: 10.0.1.1
        Area ID: 0.0.0.1
        Checksum: 0x039a [correct]
        Auth Type: Null (0)
        Auth Data (none): 0000000000000000
    LS Update Packet
        Number of LSAs: 1
        LSA-type 1 (Router-LSA), len 48
            .000 0000 0000 0010 = LS Age (seconds): 2
            0... .... .... .... = Do Not Age Flag: 0
            Options: 0x22 ((DC) Demand Circuits, (E) External Routing)
                0... .... = DN: Not set
                .0.. .... = O: Not set
                ..1. .... = (DC) Demand Circuits: Supported
                ...0 .... = (L) LLS Data block: Not Present
                .... 0... = (N) NSSA: Not supported
                .... .0.. = (MC) Multicast: Not capable
                .... ..1. = (E) External Routing: Capable
                .... ...0 = (MT) Multi-Topology Routing: No
            LS Type: Router-LSA (1)
            Link State ID: 10.0.3.3
            Advertising Router: 10.0.3.3
            Sequence Number: 0x80000006
            Checksum: 0xfcc3
            Length: 48
            Flags: 0x00
                .... .0.. = (V) Virtual link endpoint: No
                .... ..0. = (E) AS boundary router: No
                .... ...0 = (B) Area border router: No
            Number of Links: 2
            Type: Transit  ID: 10.0.2.1        Data: 10.0.2.3        Metric: 1
                Link ID: 10.0.2.1 - IP address of Designated Router
                Link Data: 10.0.2.3
                Link Type: 2 - Connection to a transit network
                Number of Metrics: 0 - TOS
                0 Metric: 1
            Type: Transit  ID: 10.0.3.4        Data: 10.0.3.3        Metric: 1
                Link ID: 10.0.3.4 - IP address of Designated Router
                Link Data: 10.0.3.3
                Link Type: 2 - Connection to a transit network
                Number of Metrics: 0 - TOS
                0 Metric: 1
\end{lstlisting}


The OSPF part of the packet is divided into 2 parts: the header and the update packet itself. The OSPF header contains information about the used version, the message type, the packets length, the ip address of the source router, the area ID, the checksum and authentication data. \\ \\
The update packet itself contains the LSAs itself. In this packet 1 LSA is included.  The LSA type for this particular advertisement is Router-LSA, meaning it contains information about directly connected links to the router in the area of that router. As for options, demand circuits and external routing is specified. Demand circuits surpresses periodic hello and update packets. The external routing option means that the source router is capable of accepting external LSAs. \\
After the options, the Link State ID and Advertising Router are specified (in this case $10.0.3.3$). Following this we have the sequence number, the checksum and the length of the packet. \\
Lastly, the connected links are specified. The link ID, link data, link type, number of metrics and metric are specified for each link. The link ID and link data have different meanings depending on the link type. In our case, for link type 2, the link ID is the IP address of the Designated Router. The link data is the IP address of the interface (to the network) from the Advertising Router (e.g. for the first link: interface vlan1 from Router1). The metric for the links is a cost calculated by ospf, depending on the bandwidth of the interface.