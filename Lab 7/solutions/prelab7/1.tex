NAT is often mentioned as a solution to counteract the depletion of IP addresses on the global internet as it can translate global IP + port data to local ip addresses and ports. The most common usage of this is the typical SOHO setup of having a home network with a single access point to the global internet (and therefore a single IP address) through the router. In order for devices on this home network to communicate with the global internet, the router must therefore do NAT. Alternatives to NAT are the use of variable netmask sizes and IPv6.