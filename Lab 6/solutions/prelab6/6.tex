First of all, root election happens: all involved switches will set their leader id to their own id and send their id across the network. If a switch receives an id, it compares it to its current leader id. If the id is lower than the current leader, the current leader id gets replaced by the newly received id. Then all non-root bridges must elect a root port. STP assigns costs to each connected port, based on the bandwith the link supports. Then the cumulative cost for each bridge is calculated in the root bridge: the root bridge sends out BPDU's (Bridge Protocol Data Units) which accumulate the cumulative cost to each bridge. The port election is then done by selecting the port with the least cumulative cost towards the root. All the others are set to blocking.