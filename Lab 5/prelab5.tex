%!TEX root = labo.tex

\subsubsection*{TCP and UDP}
Use the following resources to prepare yourself for this lab session:
\begin{enumerate}
	\item TCP and UDP: Read the overview of TCP and UDP available at \url{http://en.wikipedia.org/wiki/Transmission_Control_Protocol} and \url{http://en.wikipedia.org/wiki/User_Datagram_Protocol}.
	\item IP Fragmentation: Refer to the website \\ \url{http://www.tcpipguide.com/free/t_IPMessageFragmentationProcess.htm} for information on IP Fragmentation and Path MTU Discovery.
	\item TCP Retransmissions: Refer to RFC 2988, which is available at \url{http://tools.ietf.org/html/rfc2988},
and read about TCP retransmissions.
	\item TCP Congestion Control: Refer RFC 2001, which is available at \url{http://tools.ietf.org/html/rfc2001},
and read about TCP congestion control.
\end{enumerate}

\newpage
\subsection*{Prelab Questions}
\begin{questions}
	\q{1}{Explain the role of port numbers in TCP and UDP.}
	\q{2}{Provide the syntax of the \cmd{ttcp} command for both the sender and receiver, which executes the following scenario: A TCP server has IP address 10.0.2.6 and a TCP client has IP address 10.0.2.7. The TCP server is waiting on port number 2222 for a connection request. The client connects to the server and transmits 2000 bytes to the server, which are sent as 4 write operations of 500 bytes each.}
	\q{3.a}{How does TCP decide the maximum size of a TCP segment?}
	\q{3.b}{How does UDP decide the maximum size of a UDP datagram?}
	\q{3.c}{What is the ICMP error generated by a router when it needs to fragment a datagram with the DF bit set? Is the MTU of the interface that caused the fragmentation also returned?}
	\q{3.d}{Explain why a TCP connection over an Ethernet segment never runs into problems with fragmentation.}
	\q{4}{Assume a TCP sender receives an acknowledgement (ACK), that is, a TCP segment with the ACK flag set, where the acknowledgement number is set to 34567 and the window size is set to 2048. Which sequence numbers can the sender transmit?}
	\q{5.a}{Describe Nagle's algorithm and explain why it is used in TCP}
	\q{5.b}{Describe Karn's Algorithm and explain why it is used in TCP}
	\q{6.a}{What is a delayed acknowledgement in TCP?}
	\q{6.b}{What is a piggybacked acknowledgement in TCP?}
	\q{7}{Describe how the retransmission timeout (RTO) value is determined in TCP.}
	\q{8.a}{Describe the sliding window flow control mechanism used in TCP .}
	\q{8.b}{Describe the concepts of slow start and congestion avoidance in TCP.}
	\q{8.c}{Explain the concept of fast retransmit and fast recovery in TCP.}
\end{questions}
